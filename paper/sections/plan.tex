\section{今後の計画}
本稿では,手法2の実装を完了することができなかった.手法2を実装し,ICE over QUICが実現すれば,手法1は元より先行事例で示した他プロトコルでのNAT越えやUDP上での純粋なICEプロトコルとの定量的な比較が可能になる.手法1との比較については特に,3章で述べたように論理的には手法2の方が通信確立は効率的だが,実際に実験を行なって速度を確認することができる.

手法2の実装と併せて,QUIC上の他プロトコルの実装も今後の計画の一つである.今回aioiceとaioquicを拡張したが,その際に両ライブラリとPythonの非同期処理に関する知識を多く得ることができた.これを生かしてQUICやP2P通信に関するプロトコルの実装,議論に参加していく.

\section{結論}
QUIC上でP2P通信を確立する手法はこれまで仕様として提案されておらず,Using QUIC to traverse NATsで初めて提案された.しかし提案された手法は実装が公開されていなかったため,QUICプロトコルへの理解,実装力の向上を目的として実装を行なった.最終的に提案手法に完全に沿った実装は間に合わなかったが、QUIC上でのP2P通信は確認することができた.提案されている2つの手法の比較や定量的な評価は今後の計画とする.
