\section{背景}
% 近年,QUIC\cite{rfc9000}の普及により,TCPやUDP上で動作する様々なアプリケーションをQUIC上に移行するニーズが増加している.ビデオ会議やゲームなどP2P通信を利用するアプリケーションは接続の継続性を持つQUICの恩恵を受けやすく,特に移行によるメリットが大きい.しかし複数ホスト間のP2P通信を確立するためにはパケットがNATデバイスを通過するための仕組み,NAT越えが必要であり,プロトコルごとにNAT越えの手法は異なる.

% TCP,UDP上でのNAT越えやP2P通信を包括的に管理するプロトコルはすでに定義され,多くの実装が存在する.しかしQUIC上でのNAT越えを定義するプロトコルは存在せず,Using QUIC to traverse NATs\cite{quic_nat}で初めて手法が二つ提案されたが未だ実装は公開されていない.

% 本稿では,将来的にこのプロトコルやQUICの上に乗る他プロトコルの議論に関わる実装力をつける目的で,Using QUIC to traverse NATsで提案されている二つの手法を実装する.

ビデオ会議やオンラインゲームなどのサービスにおいて,クライアント・サーバー型通信と比較して遅延の少ないP2P通信は広く用いられており,そのほとんどは通信プロトコルにUDP\cite{rfc768}を用いている.また2021年5月にQUIC\cite{rfc9000}がRFC9000として公開され,HTTP/3\cite{rfc9114}での利用をはじめとして多くのプロトコルやサービスの根幹技術として普及している.

UDPと比較してQUIC上でP2P通信を確立することはアプリケーションに以下のメリットをもたらす.
\begin{itemize}
    \item 多重ストリーム機能と輻輳制御によるより安定したデータ転送
    \item コネクションマイグレーション機能を利用したネットワーク切り替え時の接続維持
\end{itemize}

これらのメリットを活用するために,IETF quic WGでは,QUIC上でP2P通信を確立する手法を提案するドラフトUsing QUIC to traverse NATs\cite{quic_nat}が提案され,議論されている.本稿執筆時点でこのドラフトでは二つの手法が提案されているが,そのどちらの実装も公開されていない.

そこで本稿では,筆者のQUICへの理解,実装力の向上とQUICと従来のプロトコルでのP2P通信の比較を目的として,Using QUIC to traverse NATsで提案されている二つの手法の実装を行う.
